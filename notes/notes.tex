\documentclass[a4paper,12pt]{article}

\usepackage{times}
\usepackage{amsmath}

\begin{document}

(Notes on some details of implementation for reference during development)


Local correction integrals are implemented as matrix multiplications
of the source vector at patch nodes:
\begin{align}
  \phi(\mathbf{x}_{i})
  &=
  \sum_{j} A_{ij}\sigma_{j},
\end{align}
where $A_{ij}$ is a matrix of size $N_{n}\times n_{p}$, with $N_{n}$
the number of neighbours, including the patch nodes, and $n_{p}$ the
number of patch nodes. The result of the matrix multiplication is
given at nodes $\mathbf{x}_{i}$, the location of the $i$th entry in
the neighbour list of the patch. The correction matrices are computed
for the single and double layer potentials and packed together. 

\end{document}

%%% Local Variables:
%%% mode: latex
%%% TeX-master: t
%%% End:
