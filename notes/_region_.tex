\message{ !name(notes.tex)}\documentclass[a4paper,12pt]{article}

\usepackage{times}
\usepackage{amsmath}

\newcommand{\D}{\mathrm{d}}
\newcommand{\E}{\mathrm{e}}
\newcommand{\J}{\mathrm{j}}
\newcommand{\imag}{\mathcal{I}}
\newcommand{\real}{\mathcal{R}}

\bibliographystyle{unsrt}

\begin{document}

\message{ !name(notes.tex) !offset(-3) }


(Notes on some details of implementation for reference during development)


Local correction integrals are implemented as matrix multiplications
of the source vector at patch nodes:
\begin{align}
  \phi(\mathbf{x}_{i})
  &=
  \sum_{j} A_{ij}\sigma_{j},
\end{align}
where $A_{ij}$ is a matrix of size $N_{n}\times n_{p}$, with $N_{n}$
the number of neighbours, including the patch nodes, and $n_{p}$ the
number of patch nodes. The result of the matrix multiplication is
given at nodes $\mathbf{x}_{i}$, the location of the $i$th entry in
the neighbour list of the patch. The correction matrices are computed
for the single and double layer potentials and packed together. 

\begin{table}
  \centering
  \begin{tabular}{lll}
    \verb+laplace_G+  & $G_{L}$ & \\
    \verb+laplace_dG+ & $\partial G_{L}/\partial n$ & evaluated for
    normal at field point\\
    \verb+helmholtz_Gr+ & $\real (G_{H})$ & \\
    \verb+helmholtz_Gi+ & $\imag (G_{H})$ & \\
    \verb+helmholtz_dGr+ & $\real (\partial G_{H}/\partial n)$ &
    evaluated for normal at field point\\
    \verb+helmholtz_dGi+ & $\imag (\partial G_{H}/\partial n)$ &
    evaluated for normal at field point
  \end{tabular}
  \caption{Source terms for generation of boundary conditions; syntax
    for expressions is given by the help function of the solver}
  \label{tab:sources}
\end{table}

\begin{align}
  G_{L}(\mathbf{x}) &= \frac{1}{4\pi R},\\
  G_{H}(\mathbf{x}) &= \frac{\E{^{\J k R}}}{4\pi R},\\
  R &= |
\end{align}

% \begin{align}
%   G(\mathbf{x}) &= \frac{1}{4\pi R},\quad
% \end{align}

\bibliography{abbrev,scattering}

\end{document}

%%% Local Variables:
%%% mode: latex
%%% TeX-master: t
%%% End:

\message{ !name(notes.tex) !offset(-73) }
